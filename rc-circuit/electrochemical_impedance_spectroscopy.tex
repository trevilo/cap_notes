\section{Electrochemical impedance spectroscopy}

\begin{figure}
    \centering
    \begin{circuitikz}
        \draw (-0.5,0)
        to[short,o-] (0,0)
        to[generic,l=Z,v=U] (2,0)
        to[short,i=I] (2.5,0)
        to[short,-o] (3,0);
    \end{circuitikz}
    \caption{Electrical impedance.
    }
\end{figure}
By definition, the impedance $\text{Z}$ is the measure of the opposition that
an electrical element or circuit presents to a current when a voltage is
applied.  It is the frequency domain ratio of the voltage to the current
\begin{equation}
    \text{Z} = \frac{\text{U}}{\text{I}}
\end{equation}
It is a complex number.
\begin{equation}
    \text{Z} = |\text{Z}| e^{j \arg(\text{Z})}
    = \text{R} + j \text{X}
\end{equation}
The magnitude of the impedance $|\text{Z}|$ acts just like a resistance.  It
gives the drop in voltage amplitude across an impedance $\text{Z}$ for a given
current $\text{I}$.  The phase $\theta = \arg(\text{Z})$ tells by how much the
current lags the voltage.
The real part $\text{R} = |\text{Z}| \cos \theta$ and imaginary part
$\text{X} = |\text{Z}| \sin \theta$ of the impedance are called resistance and
reactance, respectively.  They are both measured in ohms ($\ohm$).
If $\text{X} > 0$, the reactance is said to be inductive.
If $\text{X} < 0$, the reactance is capacitive.
If $\text{X} = 0$, then the impedance is purely resistive.  
An inductor has a purely reactive impedance proportional to the signal
frequency $f$ and inductance $\text{L}$, $\text{X}_\text{L} = 2\pi f \text{L}$.
The impedance of a capacitor is inversely proportional to the frequency $f$
and capacitance $\text{C}$, $\text{X}_\text{C} = \frac{1}{2\pi f \text{C}}$.



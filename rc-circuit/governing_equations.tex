%%%%%%%%%%%%%%%%%%%%%%%%%%%%%%%%%%%%%%%%%%%%%%%%%%%%%%%%%%%%%%%%%%%%%%%%%%%%%%
\subsection{Governing equations}
%%%%%%%%%%%%%%%%%%%%%%%%%%%%%%%%%%%%%%%%%%%%%%%%%%%%%%%%%%%%%%%%%%%%%%%%%%%%%%
Ignoring the influence of the electrolyte concentration, the current density
in the matrix and solution phases can be expressed by Ohm's law as
\begin{align}
    i_1 &= - \sigma \nabla \Phi_1 \\
    i_2 &= - \kappa \nabla \Phi_2
\end{align}
$i$ and $\Phi$ represent current density and potential; subscript indices
1 and 2 denote respectively the solid and the liquid phases.
$\sigma$ and $\kappa$ are the matrix and solution phase conductivities.

The total current density is given by $i = i_1 + i_2$.  Conservation of
charge dictates that
\begin{equation}
    -\nabla \cdot i_1 = \nabla \cdot i_2 = a i_n
\end{equation}
where $a$ is the interfacial area per unit volume and the current transferred
from the matrix phase to the electrolyte $i_n$ is the sum of the double-layer
the faradaic currents
\begin{align}
    i_n %= i_{n,dl} + i_{n,f}
    = C \frac{\partial}{\partial t} (\Phi_1 - \Phi_2)
    + i_0 \left(
        \exp  \frac{\alpha_a F}{RT} \eta
        -
        \exp -\frac{\alpha_c F}{RT} \eta
    \right)
\end{align}
$C$ is the double-layer capacitance.  $i_0$ is the exchange current density,
$\alpha_a$ and $\alpha_c$ the anodic and cathodic charge transfer
coefficients, respectively.
$F$, $R$, and $T$ stand for Faraday's constant, the universal gas constant and
temperature.
$\eta$ is the overpotential relative to the equilibrium potential $U_{eq}$
\begin{equation}
    \eta = \Phi_1  - \Phi_2 - U_{eq}
\end{equation}

%%%%%%%%%%%%%%%%%%%%%%%%%%%%%%%%%%%%%%%%%%%%%%%%%%%%%%%%%%%%%%%%%%%%%%%%%%%%%%
\subsubsection{Note}
%%%%%%%%%%%%%%%%%%%%%%%%%%%%%%%%%%%%%%%%%%%%%%%%%%%%%%%%%%%%%%%%%%%%%%%%%%%%%%
For simplicity, we consider for now a linearized version of the Butler-Volmer
kinetics
\begin{equation}
    i_{n,f} \approx 
        i_0 \frac{(\alpha_a + \alpha_c) F}{RT} \left( \Phi_1 - \Phi_2 \right)
\end{equation}
This is true for overpotentials with absolute value that does not exceed
$RT/F$, which is approximatively 0.05 V at ambient temperature.
Admittedly, we will use it beyond its validity range.

$U_{eq}$ is a measure of the state of charge (it ``shifts'' the equilibrium).
For now, we assume it is zero, which makes the supercapacitor behave similarly
to a parallel resistor capacitor model.  At open circuit, the capacitor slowly
discharges back to 0 V.


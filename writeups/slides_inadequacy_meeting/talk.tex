\documentclass[10pt,xcolor=dvipsnames,compress]{beamer}

\usepackage{danny_theme}
\usepackage{danny_style}
\usepackage{hyperref}
%\usepackage[algoruled,longend]{algorithm2e}
\usepackage{array}
\usepackage{graphicx}
\usepackage{amsmath}
\newcolumntype{C}{>{\centering\arraybackslash} m{1.3in} }
\newcolumntype{M}{>{\centering\arraybackslash} m{1.in} }
\everymath{\displaystyle}
\long\def\/*#1*/{}

\beamertemplatenavigationsymbolsempty


%===============================================================================
% 					  Presentation Title and Author  
%===============================================================================
\title[Supercapacitor]{
Inadequecy Representation of Supercapacitor Batteries Models}
%\subtitle{Nonlinear Deformation in Metallic Materials}

\author[Danial Faghihi]{Danial Faghihi}

\institute[ICES]{Institute for Computational Engineering and Sciences (ICES)\\
$\quad~$The University of Texas at Austin
}

\date[\today]{\today}
%===============================================================================
%===============================================================================



\begin{document}

%===============================================================================
% SLIDE 00
%===============================================================================
\begin{frame}
\titlepage
\end{frame}

%===============================================================================
% SLIDE 00
%===============================================================================
\begin{frame}
\frametitle{Outline}
\vfill

\vspace{0.7in}
\tableofcontents
\vspace{0.7in}

\vfill
\end{frame}




%%%%%%%--------------------------------------------------------------------------------------------------------------------------
\section{Motivation}
%%%%%%%--------------------------------------------------------------------------------------------------------------------------
%===============================================================================
% SLIDE 00
%===============================================================================
\begin{frame}
\frametitle{Outline}
\vfill

\vspace{0.7in}
\tableofcontents[currentsection,currentsubsection] 
\vspace{0.7in}

\vfill
\end{frame}


%===============================================================================
% SLIDE 01
%===============================================================================
\begin{frame}
\frametitle{What are supercapacitors?}
\vfill


Supercapacitors are intermediate power/energy storage/supply devices that bridge the gap between \textit{electrolytic capacitors} and  \textit{rechargeable batteries}. They can provide
\begin{itemize}
\item higher energy density (capacitance) than capacitors
\item higher power density (faster charge delivery) than batteries
\item many more charge and discharge cycles than batteries
\end{itemize}

\begin{center}
\vspace{-0.06in}
\includegraphics[trim = 0mm 0mm 0mm 0mm, clip, width=.48\textwidth]{figs/cap_battery}
\vspace{-0.1in}
\end{center}

\begin{small}
Supercapacitors are suitable in applications where a large amount of power is needed for a relatively
short time, where a very high number of charge/discharge cycles or a longer lifetime is required.
e.g. Low supply current for memory backup in SRAM, power for cars, etc.
\end{small}

\vfill
\end{frame}


%===============================================================================
% SLIDE 02
%===============================================================================
\begin{frame}
\frametitle{What are supercapacitors?}
\vfill


\begin{columns}[T] % contents are top vertically aligned
     \begin{column}[T]{6cm} % each column can also be its own environment
	\includegraphics[trim = 0mm 0mm 0mm 0mm, clip, width=.7\textwidth]{figs/stacked.png}
	\\Supercapacitor with stacked electrodes
     \end{column}
     \begin{column}[T]{4cm} % alternative top-align that's better for graphics
	\includegraphics[trim = 0mm 0mm 0mm 0mm, clip, width=.7\textwidth]{figs/wound.png}
	\\ Wound supercapacitor
     \end{column}
\end{columns}

\begin{block}{}
\begin{columns}[T] % contents are top vertically aligned
     \begin{column}[T]{3cm} % each column can also be its own environment
\includegraphics[trim = 2.4in 0.42in 0.7in 0.9in, clip, width=1\textwidth]{figs/supercap_schematic.pdf}  
    \end{column}
     \begin{column}[T]{7cm} % alternative top-align that's better for graphics
Unit cell:
\begin{itemize}
\begin{small}
\item Anode current collector
\item Porous anode electrode: solid matrix filled with liquid electrolyte
\item Separator: electronic insulator and ion permeable
\item Porous cathode electrode: solid matrix and liquid electrolyte
\item Cathode current collector.
\end{small}
\end{itemize}
     \end{column}
\end{columns}
\end{block}

\vfill
\end{frame}


%===============================================================================
% SLIDE 02
%===============================================================================
\begin{frame}
\frametitle{Storage principles}
\vfill

Capacitance value of an electrochemical capacitor is determined by two storage principles
\begin{itemize}
\item double-layer capacitance: electrostatic storage of the electrical energy
by separation of charge in a double layer at the interface between electrode/electrolyte .
The amount of electric charge stored is linearly proportional to the applied voltage and depends primarily on the electrode surface.
\item pseudo capacitance: electrochemical storage achieved by faradaic redox reactions with charge-transfer.
\end{itemize}

\begin{block}{}
explanations!!!
\end{block}


\vfill
\end{frame}


%%%%%%%--------------------------------------------------------------------------------------------------------------------------
\section{Model Description}
%%%%%%%--------------------------------------------------------------------------------------------------------------------------
%===============================================================================
% SLIDE 00
%===============================================================================
\begin{frame}
\frametitle{Outline}
\vfill

\vspace{0.7in}
\tableofcontents[currentsection,currentsubsection] 
\vspace{0.7in}

\vfill
\end{frame}


%===============================================================================
% SLIDE 03
%===============================================================================
\begin{frame}
\frametitle{Governing Equations}
\vfill

\begin{block}{}
Current density following Ohm's law:
\begin{itemize}
\item Electrode (matrix phase) : due to electrons migration
$
\mathbf{i}_1 = -\sigma\nabla\phi_1
$
\item Electrolyte (solution phase) : due to ion migration
$
\mathbf{i}_2 = -\kappa\nabla\phi_2
$
\end{itemize}
where $\phi_1$ and $\phi_2$ are potentials, and 
$\sigma$ is solid matrix electronic conductivity and $\kappa$ is liquid ionic conductivity.
\end{block}

\begin{block}{conservation of charge}
\vspace{-.2in}
\begin{eqnarray*}
{\rm total \; current \; density:} \; I & = & \mathbf{i}_1 + \mathbf{i}_2\\
-\nabla \cdot \mathbf{i}_1 & = & \nabla\cdot \mathbf{i}_2 = a {i}_n
\vspace{-.3in}
\end{eqnarray*}
$a$: interfacial area per unit volume \\
${i}_n$: current transferred from the matrix to the electrolyte
\begin{equation*}
{i}_n = \underbrace{C \frac{\partial}{\partial t} \eta}_{\rm double-layer}+
\underbrace{{i}_0 ( \exp (\frac{\alpha_aF}{RT}\eta) - \exp (-\frac{\alpha_aF}{RT}\eta)}_{\rm faradaic})
\end{equation*}
overpotential: 
$
\eta =  \phi_1 - \phi_2.
$
\end{block}

\vfill
\end{frame}


%===============================================================================
% Slide 04
%===============================================================================
\begin{frame}
\frametitle{High Fidelity Model}
\vfill

\begin{columns}
\begin{column}{.47\textwidth}
\begin{itemize}
\begin{small}
\item $\eta(\xi,\tau)$ = overpotential in electrode\\
\item $\gamma = \frac{\kappa}{\sigma}$ : conductivity ratio \\
\item $\xi, \tau$ : dimensionless distance/time \\
\item $I(\tau)$ : dimensionless current
\end{small}
\end{itemize}
\end{column}
%-----------------------------
\begin{column}{.47\textwidth} 
	\includegraphics[trim = 0in 1.4in 0in 1.4in, clip, width=1\textwidth]{figs/schematic.pdf}
\end{column}
\end{columns}
%%====================
\begin{columns}
\begin{column}{.47\textwidth}
%-----------------------------
\begin{problock}{High Fidelity (1D) model}
\begin{equation*}\label{eq:HF}
\frac{\partial\eta_{HF}}{\partial\tau} = \frac{\partial^2\eta_{HF}}{\partial\xi^2}
\end{equation*}
\begin{equation*}
\left\{\begin{matrix}
\frac{\partial\eta_{HF}}{\partial\xi}|_{\xi=0} & = & -\frac{\gamma}{1+\gamma}I(\tau)\\
\frac{\partial\eta_{HF}}{\partial\xi}|_{\xi=1} & = & \frac{1}{1+\gamma}I(\tau) \nonumber\\
\eta_{HF}|_{\tau=0} 					       & =  & \eta_0(\xi)
\end{matrix}\right.
\end{equation*}
\end{problock}
\end{column}
%-----------------------------
\begin{column}{.47\textwidth}
\vspace{-0.1in}
\begin{alertblock}{Modeling Assumptions (Sins)}
\begin{enumerate}[i.]

\begin{footnotesize}
\item No Faradaic processes:
current transferred from matrix to the solution phase goes towards only charging the double-layer at the electrode/electrolyte interface.

\item $\phi_1$ is uniformly distributed over the current collector domain (collector is sufficiently thin)
%
%The electrical resistivity of the current collector is low (or it is sufficiently thin) that one can assume uniform distribution of $\phi_1$ over the collector domain: \textit{homogeneous in the $x$-direction}, 

\item There is no electron/ion fluxes cross the top and bottom boundaries
%. Also due to high conductivity of collectors, the voltage over the whole interface on the collector side is negligible (similar to case that tab on the left collector is grounded): \textit{2D domain could be reduced to a quasi-1D domain},

\item The material properties are constant within each layer
\end{footnotesize}

\end{enumerate}
\end{alertblock}
\end{column}
\end{columns}


\vfill
\end{frame}


%===============================================================================
% Slide 05
%===============================================================================
\begin{frame}
\frametitle{Low Fidelity Model}
\vfill


\begin{alertblock}{Modeling Assumptions (Sins)}
\begin{enumerate}[i.]

\item Spatially
averaging the governing equation over the entire domain length (PDE of the HF model reduces to an ODE) 
%
\begin{equation}\label{eq:LF_avg}
\eta^{avg} = \int_0^1 \eta d\xi \qquad \Rightarrow \qquad
\frac{\partial{\eta}^{avg}}{\partial\tau} = I^*
\end{equation}

\item Assuming a quadratically varying profile for overpotential inside the electrodes
\begin{equation*}\label{eq:quadratic}
\eta_{LF} (\xi,\tau)= a(\tau)\xi^2 + b(\tau)\xi + c(\tau)
\end{equation*}
where $a$, $b$, and $c$ can be obtained from PDE+BCs of HF model.

\end{enumerate}
\end{alertblock}

%-----------------------------
\begin{block}{Low Fidelity (0D) model}
\begin{equation*}\label{eq:LF}
\eta_{LF}(\xi,\tau) = 
\frac{1}{2}I^*(\tau)\xi^2 - I^*(\tau) \frac{\gamma}{1+\gamma}\xi + {\eta}^{avg}(\tau) - \frac{I^*(\tau)}{6} + \frac{I^*(\tau)}{2}\frac{\gamma}{1+\gamma}
\end{equation*}
${\eta}^{avg}$ is the solution of (\ref{eq:LF_avg}) given appropriate initial condition.
\end{block}




\vfill
\end{frame}



%===============================================================================
% Slide 06
%===============================================================================
\begin{frame}
\frametitle{QoI : cell voltage }
\vfill


\begin{alertblock}{Quantity of Interest}
 Potential drop across the system

 $V^{\rm cell}(\tau) = \phi_{\rm col.}^L - \phi_{\rm col.}^R
= \frac{1+2\gamma}{1+\gamma}\eta|_{\xi=1} - \frac{\gamma}{1+\gamma}\eta|_{\xi=0} - \frac{\gamma}{(1+\gamma)^2}I
 $
\end{alertblock} 


\begin{figure}[h]
    \centering
    \includegraphics[trim = 1.2in 2.4in 1.6in 2.8in, clip, width=0.45\textwidth]{figs/I.png}
    \\
    \includegraphics[trim = 1.2in 2.4in 1.6in 2.8in, clip, width=0.45\textwidth]{figs/etaLF_HF.png}   
    \\
    \includegraphics[trim = 1.2in 2.4in 1.6in 2.8in, clip, width=0.45\textwidth]{figs/Vcell_HF_LF.png} 
\end{figure}





\vfill
\end{frame}





\end{document}
